\documentclass[]{ctexart}
\usepackage{amsmath,amssymb,graphicx,textcomp,subfigure,indentfirst,ctex,color,float}
\usepackage{hyperref}
\hypersetup{hidelinks,
	colorlinks=true,
	allcolors=black,
	pdfstartview=Fit,
	breaklinks=true}

\title{Title}
\author{Siyi Zhao}
\date{\today}

\begin{document}
\maketitle

\tableofcontents

\section*{Problem Set}

\textbf{Problem 2. “Sandage-Loeb test”.} (3 credits)

The so-called “Sandage-Loeb test” offers a direct measurement of the expansion
rate of the Universe. For a light ray that was emitted at emission time $t_e$ and later observed at time $t_0$, the redshift
$$1+z=\frac{a\left(t_{0}\right)}{a\left(t_{e}\right)},$$
where $a\left(t\right)$ is the scale factor at the time $t$. If an observer keeps observing for a (\emph{very large}) duration of time $dt_0$, then there will be a change of observed redshift $dz$.
\\(1) Derive the change rate $d z / d t_{0}$  in terms of $H_0$, $z$, and Hubble parameter $H(z)$. [Hint: as the time of observation $t_0$ changes to $t_0 + dt_0$, we observe light that was also emitted at a later time $t_e + dt_e$. What is the relation of $dt_e$ and $dt_0$? You may consider a wavelet of light rays with the length $c\ dt_e$ at the time of emission. It will be redshifted to the length $c\ dt_0$ at the time of observation.]
\\(2) In the $\Lambda$CDM model, the Hubble parameter is
$$
H(z)=H_{0} \sqrt{\Omega_{\Lambda}+\Omega_{m}(1+z)^{3}++\Omega_{K}(1+z)^{2}++\Omega_{\gamma}(1+z)^{4}} .
$$ 
Assume $\Omega_{K}=0$, $\Omega_{\gamma} \approx 0$, $\Omega_{m}=0.3$, $\Omega_{\Lambda}=0.7$ and $H_{0}=70 \mathrm{~km} / \mathrm{s} / \mathrm{Mpc}$. You
observe a galaxy at $z = 1$. Determine how long (in years) you will have to keep observing the galaxy in order to see its redshift change $|d z|=10^{-6}$. Is $d z<0$  or $d z>0$ ?

\section{宇宙成分组成}

\subsection{who}

\subsubsection{哈勃参数}


$$\begin{aligned} \vec{v}(t)=\frac{d \vec{x}}{d t} &=x\left(t_{0}\right) \frac{1}{a\left(t_{0}\right)} d a / d t \\ &=x\left(t_{0}\right) \frac{a(t)}{a\left(t_{0}\right)} \frac{d a / d t}{a(t)} \end{aligned}$$

定义哈勃参数 $H(t) \equiv \frac{da/dt}{a} = \frac{\dot{a}}{a}$,可以得到哈勃定律 $\vec{V}(t)=\vec{x}(t)H(t)$,所以 $H_0$ 就是今天的哈勃参数。


因为历史原因, $H_0 = 100 h^{-1} \mathrm{~km} \mathrm{~s}^{-1} \mathrm{~Mpc}^{-1}$ , $h \simeq 0.7$ ,更精确的测量出现 Hubble tension 问题。

\begin{itemize}
    \item 
\end{itemize}

\begin{equation} \label{eq}
    \mathrm{GeV} M_{\mathrm{Pl}} \left\langle \hat{a}^{\dagger} \right\rangle \{1\cdots n\} a \implies b   \ddot{a} \to 0 
\end{equation}

\begin{equation*}
    \hbar \notin \mathbb{R} 
\end{equation*}

\ref{eq}

\href{http://astro.tsinghua.edu.cn/~wzhu/student-seminar-2022/index.html}{2022 Spring student seminar}



\section{PNG21cm}

\subsection{bias meaurement}

\begin{equation}
    \delta_{x_\mathrm{HI}}(\boldsymbol{x}) = \frac{x_\mathrm{HI}(\boldsymbol{x})}{\bar{x}_\mathrm{HI}} - 1
\end{equation}

\begin{equation}
    \delta_{\rho_{\mathrm{HI}}}(\boldsymbol{x}) = \delta_m(\boldsymbol{x}) + \delta_{x_\mathrm{HI}}(\boldsymbol{x}) + \delta_m(\boldsymbol{x}) \delta_{x_\mathrm{HI}}(\boldsymbol{x})
\end{equation}

\begin{equation}
    H = \frac{L_\mathrm{box}}{N_\mathrm{cell}}
\end{equation}

\begin{equation*}
    \delta_{\rho_{\mathrm{HI}}}(\boldsymbol{x})=b_{1} \delta_{m}(\boldsymbol{x})+\frac{1}{2} b_{2} \delta_{m}^{2}(\boldsymbol{x})
\end{equation*}



\subsection{Theoritical prediction of bispectrum}

\begin{eqnarray}
    x_\mathrm{HII} &=& 1 - x_\mathrm{HI} \\ 
    b_\mathrm{HII}^\mathrm{G} &=& \frac{1 - b_1 x_\mathrm{HI} }{x_\mathrm{HII}} 
\end{eqnarray}







\section{“星系与宇宙”助教笔记}

\subsection{有趣的问题}

Q1:为什么宇宙的年龄是哈勃常数的倒数?
\begin{itemize}
    \item 解弗里德曼方程,
    \begin{itemize}
        \item 物质主导的宇宙解出 $a \propto t^\frac{2}{3}$ , 那么 $H=\frac{\dot{a}}{a}=\frac{2}{3t}$ ,在今天取值 $t_0 = \frac{2}{3 H_0}$。
        \item 辐射主导的宇宙解出 $a \propto t^\frac{1}{2}$ , 那么 $H=\frac{\dot{a}}{a}=\frac{1}{2t}$ ,在今天取值 $t_0 = \frac{1}{2 H_0}$。
        \item generally, $a \propto t^\frac{2}{3(w+1)}$ , 那么 $H=\frac{\dot{a}}{a}=\frac{2}{3(w+1) t}$ ,在今天取值 $t_0 = \frac{2}{3(w+1) H_0}$。
    \end{itemize}
    \item 定性地可以理解为:宇宙以多项式的形式膨胀 $a \propto t^n$ (在大多数时间是的,inflation的时候不是), $H=\frac{\dot{a}}{a}\propto \frac{1}{t}$ 
\end{itemize}

QN:
\begin{itemize}
    \item 牛顿力学在均匀无限大宇宙中存在矛盾:$\nabla \cdot \mathbf{g}=-4 \pi G \rho$, $\mathbf{g} \equiv 0$, while $\rho \neq 0$ 
\end{itemize}



\subsection{宇宙学部分参考书目推荐:}

\begin{itemize}
    \item Astrophysics for Physicists--Choudhuri.pdf Chp 10, 11,15(?)
\end{itemize}

% \cite{daloisio2013dbk}

% \bibliographystyle{aasjournal}
% \bibliography{main}

\end{document}
