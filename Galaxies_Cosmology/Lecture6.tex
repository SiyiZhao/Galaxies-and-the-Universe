\documentclass[12pt]{ctexart}
\usepackage{amsmath,graphicx,textcomp,subfigure,indentfirst,ctex,color,float}
\title{Lecture 6}
\author{赵思逸}

\date{\today}
\newcommand{\new}[1]{\textcolor{blue}{#1}}

\newcommand{\refeq}[1]{式~(\ref{#1})}
\newcommand{\reffig}[1]{图~(\ref{#1})}

\begin{document}

\maketitle

上节课我们说到:
最一般情况下,
Friedmann方程
\begin{equation}
    \dot{a}^2+K = \frac{8}{3} \pi G \rho a^2
\end{equation}
其中 
\begin{eqnarray}
    \rho(a) &=& \rho_R(a) + \rho_M(a) + \rho_\Lambda (a) \label{eq:rho}
    \\ &=& \rho_{R,0}\left(\frac{a}{a_0}\right)^{-4} + \rho_{M,0}\left(\frac{a}{a_0}\right)^{-3}  + \rho_{\Lambda,0}     
\end{eqnarray}
定义临界密度 $\rho_\text{crit} \equiv \frac{3H_0^2}{8\pi G}$.
则
Friedmann方程改写为 $H(t)= H_0 E(t)$,   其中 
\begin{equation}
    \frac{H}{H_0} = \sqrt{\frac{\rho}{\rho_\text{crit}}-\frac{K}{H_0^2 a^2}} =\sqrt{\Omega_R\left(\frac{a}{a_0}\right)^{-4}+\Omega_M\left(\frac{a}{a_0}\right)^{-3}+\Omega_\Lambda+\Omega_K\left(\frac{a}{a_0}\right)^{-2}}
\end{equation}
或者写成
$H(z)=H_0 E(z)$
\begin{equation}
    E(z) = \sqrt{ \Omega_R \left(1+z\right)^{4} + \Omega_M \left(1+z\right)^{3} + \Omega_\Lambda + \Omega_K \left(1+z\right)^{2} }
\end{equation}
其中我们定义了
今天各物质成分占比 $\Omega_i\equiv\frac{\rho_{i,0}}{\rho_\text{crit}}$, 
广义的物质包含冷物质、辐射、暗能量,$i=M,R,\Lambda$.
还定义了
曲率“密度” 
\begin{equation}
    \Omega_K\equiv-\frac{Kc^2}{H_0^2a_0^2}
\end{equation}
\new{这个式子}是形式上的,不是实质的物质。
\begin{itemize}
    \item 对 $K=0$, $\Omega_K=0$, $a_0$可以随意定义,一般定义为 $a_0=1$.
    \item 对 $K\neq0$, 通过选取共动坐标使得 R(共动)=1,使得$K=\pm 1$,今天的尺度因子 $a_0=$R(今天)/R(共动)= R(今天),不能随意选取 $a_0$. $|\Omega_K|=\frac{c^2}{H_0^2a_0^2}$ 可以连续变化。现有观测$|\Omega_K|< 10^{-2}\sim 10^{-3}$.
\end{itemize} 

在今天, $z=0$, $a=a_0$ , $H=H_0$,有
\begin{equation}
    \Omega_R + \Omega_M + \Omega_\Lambda + \Omega_K = 1 \label{eq:allOmega}
\end{equation}

注意最后一项 $\Omega_K$ 并不是真实的物质,只是我们用曲率定义出来的量。它决定我们对尺度因子的选取,只有当$\Omega_K = 0$时,我们才能选取$a_0=1$,当$\Omega_K\neq 0$时,
\begin{equation}
    a_0=\frac{c}{H_0\sqrt{|\Omega_K|}} 
\end{equation} 


\refeq{eq:allOmega}中,前三项才是宇宙中真实存在的物质,(见\refeq{eq:rho})它们的总和不一定是1。
不过宇宙学观测得到的结果 $\Omega_K \approx 0$,可以认为我们的宇宙近似是平坦的。(即使不平坦,宇宙的曲率也非常小。暴涨理论会对这一现象给出解释。) 
即$\rho_\text{crit}\simeq\rho_0$ ,所以$\Omega_i$也可以近似表示今天各物质成分在宇宙密度中的占比。

小结:
\begin{table}[H]
    \centering
    \begin{tabular}{|c|c|c|c|c|}
    \hline
    $\rho_0>\rho_\text{crit}$  & $\Omega>1$ & $\Omega_K<0$ & $K=+1$ & 正曲率,球面,有限无界 \\ \hline
    $\rho_0<\rho_\text{crit}$  & $\Omega<1$ & $\Omega_K>0$ & $K=-1$ & 负曲率,超球面,无限无界  \\ \hline
    $\rho_0=\rho_\text{crit}$  & $\Omega=1$ & $\Omega_K=0$ & $K=0$ & 平直,平直,无限无界 \\ \hline
    \end{tabular}
\end{table}
其中前两列是广义物质密度,第3、4列是曲率,第5列描述宇宙的时空几何。
$K=+1$时,物质主导的宇宙先膨胀后收缩;
$K=0$或$K=-1$时,物质主导的宇宙一直减速膨胀;
但有暗能量存在时,宇宙在足够晚期会进入加速膨胀。

今天的标准宇宙学模型就是要测量 $\Omega_R$, $\Omega_M$, $\Omega_\Lambda$, $\Omega_K$ 还有 $H_0$ 这些参数。
当模型确定后,就可以知道时间、距离、红移、尺度因子之间的对应关系。

\section{时间和距离}
\subsection{宇宙年龄(age)}

\begin{eqnarray}
    t_\text{age}(z) &=& \int_0^{t_\text{age}} dt' = \frac{1}{H_0}\int_0^\frac{1}{1+z} \frac{da'}{a' E(a')} \\ 
    &=& \frac{1}{H_0}\int_0^\frac{1}{1+z} \frac{da'}{a' \sqrt{\Omega_R a'^{-4}+\Omega_M a'^{-3}+\Omega_\Lambda+\Omega_K a'^{-2}}}
\end{eqnarray}
宇宙学模型 $\left(\Omega_M, \Omega_\Lambda\right) $决定 $t_\text{age}(z)$关系。因为目前测量到$\Omega_R = 2.47\times10^{-5}h^{-2}$很小,可以忽略。

举例:
\begin{itemize}
    \item 物质为主,$\Omega_M=1, \Omega_R=\Omega_\Lambda=0$, 推出 $\Omega_K=0$,\\今天 $t_0=\frac{1}{H_0}\int_0^1 \frac{da'}{a' \sqrt{a'^{-3}}}=\frac{2}{3 H_0}=9.32 \mathrm{Gyr}$ 
    \item 辐射为主,$\Omega_R=1, \Omega_M=\Omega_\Lambda=0$, 推出 $\Omega_K=0$,\\今天 $t_0=\frac{1}{H_0}\int_0^1 \frac{da'}{a' a'^{-2}}=\frac{1}{2 H_0}=6.99 \mathrm{Gyr}$ 
    \item 没有物质的空宇宙,$\Omega_R=\Omega_M=\Omega_\Lambda=0$, 推出 $\Omega_K=1$,\\今天 $t_0=\frac{1}{H_0}\int_0^1 \frac{da'}{a' a'^{-1}}=\frac{1}{ H_0}=13.98 \mathrm{Gyr}$ 
    \item $\Lambda$CDM 宇宙,$\Omega_R=0, \Omega_M=0.3,\Omega_\Lambda=0.7$, 推出 $\Omega_K=0$,\\今天 $t_0=\frac{1}{H_0}\int_0^1 \frac{da'}{a' \sqrt{0.3\times a'^{-3}+0.7} }=\frac{0.964}{ H_0}=13.47 \mathrm{Gyr}$ 。误差主要来自 $H_0$ .
\end{itemize}

\subsection{回溯时间(look-back time)}
\begin{eqnarray}
    t_\text{LB}(z)  = \frac{1}{H_0}\int_\frac{1}{1+z}^1 \frac{da'}{a' \sqrt{\Omega_R a'^{-4}+\Omega_M a'^{-3}+\Omega_\Lambda+\Omega_K a'^{-2}}}
\end{eqnarray}
宇宙学模型 $\left(\Omega_M, \Omega_\Lambda\right) $决定 $t_\text{LB}(z)$关系。

\subsection{光源与观测者在今天的距离}

定义 $\chi(z)$ 是光源与观测者在今天的距离。 

\begin{eqnarray}
    \chi(z) &=& \int_t^{t_0} cdt' \frac{a_0}{a(t')} = \frac{c}{H_0}\int_{\frac{a}{a_0}}^1 \frac{da'}{a'^2 E(a')} =\frac{c}{H_0}\int_0^z \frac{dz'}{E(z')} \\ 
    &=&  D_H \int_0^z \frac{dz'}{ \sqrt{\Omega_R(1+z')^{4}+\Omega_M(1+z')^{3}+\Omega_\Lambda+\Omega_K(1+z')^{2} }}
\end{eqnarray}

其中定义了$D_H\equiv\frac{c}{H_0}\simeq\frac{3\times10^5 \mathrm{~km} \mathrm{~s}^{-1}}{100 h \mathrm{~km} \mathrm{~s}^{-1} \mathrm{~Mpc}^{-1}} = 3000 h^{-1}  \mathrm{~Mpc}$,
宇宙学模型 $\left(\Omega_M, \Omega_\Lambda\right) $决定 $\chi(z)$关系。红移$z$是可测量量。$\chi$可以转化为光度距离或角直径距离,
\begin{eqnarray}
    d_A &=& \frac{a_0 r}{1+z}
    \\ d_L &=& (1+z) a_0 r
    \\ a_0 &=& \frac{c}{H_0\sqrt{|\Omega_K|} } = \frac{D_H}{\sqrt{|\Omega_K|}} 
\end{eqnarray}
其中 
$r = S_k(\chi/a_0)$ ,与 $H_0$ 无关。 
\begin{equation}
    S_k(x) = 
    \begin{cases}
        \sin x & K=+1 \\ 
        x & K=0 \\ 
        \sinh x & K=-1
    \end{cases}
\end{equation}

$r$ 将可测量量 $d_A$ 和 $d_L$ 与  $\chi$ 联系起来。通过测量 $d_A$ 和 $d_L$ 就可以得到 $\chi(z)$ 关系,由观测到的 $\chi(z)$关系就可以限制宇宙学模型。

\subsubsection{应用举例:Alcock-Paczynski test}

考虑有固定物理尺寸的球体(直径为$D$)在红移$z$的地方,观测到张角 $\Delta \theta$,红移宽度 $\Delta z$  
\begin{equation}
    \Delta \theta = \frac{D }{d_A} = \frac{D(1+z)}{a_0r} = \frac{D(1+z)\sqrt{|\Omega_K|}}{D_H S_k(\chi/a_0)}
\end{equation}

\begin{equation}
    D=\frac{a(z)}{a_{0}} \Delta \chi=\frac{1}{1+z} \frac{c}{H_{0}} \frac{\Delta z}{E(z)}
\end{equation}

\begin{equation}
    \frac{\Delta \theta}{\Delta z}(z)=\frac{\sqrt{\left|\Omega_{k}\right|}}{E(z) S_{k}\left(\chi / a_{0}\right)}=\frac{\sqrt{\left|\Omega_{k}\right|}}{E(z)}\left[S_k\left(\sqrt{\left|\Omega_{k}\right|} \int_{0}^{z} \frac{d z^{\prime}}{E\left(z^{\prime}\right)}\right)\right]^{-1}
\end{equation}
\textbf{与 $D$ 和 $ H_0$ 无关。}与 $\left(\Omega_M, \Omega_\Lambda\right) $有关。

在 $K=0$ 的情况下 
\begin{equation}
    \frac{\Delta \theta}{\Delta z}(z)= \left[E(z) \int_{0}^{z} \frac{d z^{\prime}}{E\left(z^{\prime}\right)}\right]^{-1}
\end{equation}

举例:
\begin{itemize}
    \item $\Lambda$CDM 宇宙,$\Omega_M=0.3,\Omega_\Lambda=0.7$, 推出 $\Omega_K=0$,在 $z=1$ 的地方,$\frac{\Delta \theta}{\Delta z}=\left[\sqrt{0.3 \times 2^{3}+0.7} \times \int_{0}^{1} \frac{d z^{\prime}}{\sqrt{0.3\left(1+z’\right)^{3}+0.7}}\right]^{-1}=0.736.$ 
    \item $\Omega_M=0, \Omega_\Lambda=1$, 推出 $\Omega_K=0$,\\ 在 $z=1$ 的地方,$\frac{\Delta \theta}{\Delta z}=\left[\sqrt{1} \times \int_{0}^{1} \frac{d z^{\prime}}{\sqrt{1}}\right]^{-1}=1.$ 
    \item $\Omega_M=1.3, \Omega_\Lambda=0$, 推出 $\Omega_K=-0.3<0, K=+1, S_k(x)=\sin x$,\\ 在 $z=1$ 的地方,\\$\frac{\Delta \theta}{\Delta z}=\frac{\sqrt{0.3}}{\sqrt{1.3\times 2^3-0.3\times 2^2}}\left[\sin(\sqrt{0.3}  \int_{0}^{1} \frac{d z^{\prime}}{\sqrt{1.3(1+z')^3 - 0.3 (1+z')^2}})\right]^{-1}=0.594.$  
    \item $\Omega_M=0.3, \Omega_\Lambda=0$, 推出 $\Omega_K=0.7>0, K=-1, S_k(x)=\sinh x$,\\ 在 $z=1$ 的地方,\\ $\frac{\Delta \theta}{\Delta z}=\frac{\sqrt{0.7}}{\sqrt{0.3\times 2^3 + 0.7\times 2^2}}\left[\sinh(\sqrt{0.7}  \int_{0}^{1} \frac{d z^{\prime}}{\sqrt{0.3(1+z')^3 + 0.7 (1+z')^2}})\right]^{-1}=0.640.$  
\end{itemize}

\section{宇宙学常数和真空能}

\subsection{真空能视角}
真空能 $P=-\rho$.

\begin{equation} 
    T^\mu_{\nu} = \left(\begin{array}{llll}-\rho & 0 & 0 & 0 \\ 0 & P & 0 & 0 \\ 0& 0& P &0 \\ 0 & 0&0&P\end{array}\right) = -\rho \delta^\mu_{\nu}
\end{equation}
得到真空能的能动量张量 $T_{\mu\nu}^{(\Lambda)}=- \rho_\Lambda g_{\mu\nu}$.

Einstein场方程 $R_{\mu\nu} - \frac{1}{2} g_{\mu\nu} R = -8\pi G T_{\mu\nu}$.
其中 $R_{\mu\nu}$ 是 Ricci tensor,  $R$ 是 Ricci scalar, 二者都是  $g_{\mu\nu}$ 及其导数的函数。

$T_{\mu\nu}$ 是能动量张量,包括物质和辐射(下式第一项)和真空能(下式第二项)

\begin{equation}
    T_{\mu\nu} = T_{\mu\nu}^{(M)} + T_{\mu\nu}^{(\Lambda)} = T_{\mu\nu}^{(M)} - \rho_\Lambda g_{\mu\nu}
\end{equation}

此时场方程变成 
\begin{equation} \label{Eeq.vacuum}
    R_{\mu\nu} - \frac{1}{2} g_{\mu\nu} R = -8\pi G T_{\mu\nu} =  -8\pi G T_{\mu\nu}^{(M)} +  8\pi G \rho_\Lambda g_{\mu\nu}
\end{equation}

\subsection{宇宙学常数视角}
Einstein场方程是由作用量导出的。作用量$S$为

\begin{equation}
    S = \int d^4 x \sqrt{-g} \mathcal{L} 
\end{equation}
其中 $d^4 x \sqrt{-g}$ 是4维协变的体积元,$\mathcal{L}$ 是拉氏量密度,要求 
\begin{itemize}
    \item[1.] 是4维坐标变换下的不变量。
    \item[2.] 包含度规的最高2阶导数。
\end{itemize}

只有Ricci scalar同时满足这两个条件,还有一个平凡(trivial)解——常数。 把宇宙学常数加到作用量里:
\begin{equation}
    S = \int d^4 x \sqrt{-g} \left( R  + \Lambda \right) 
\end{equation}
导出的场方程为
\begin{equation}
    R_{\mu\nu} - \frac{1}{2} g_{\mu\nu} R - \Lambda g_{\mu\nu} =  -8\pi G T_{\mu\nu}^{(M)} 
\end{equation}

与 \refeq{Eeq.vacuum} 相比,可得
\begin{equation}
    \Lambda = 8\pi G \rho_\Lambda 
\end{equation}

真空能与宇宙学常数是等价的。只不过真空能的引入有一些量子场论中的动机,而宇宙学常数则来自对广义相对论的Einstein场方程理论上的推广。下面我们将混用“真空能”和“宇宙学常数”。

\subsection{Einstein静态宇宙模型}

静态宇宙模型要求 $a$ 为常数,$\dot{a}=\ddot{a}=0$,带入弗里德曼方程得到
\begin{eqnarray}
    \rho+3P&=&0 \\
    K &=& \frac{8}{3} \pi G \rho a^2
\end{eqnarray}

如果只有物质和辐射,$\rho+3P>0$,所以Einstein在1917年引入了宇宙学常数。以下推导使用真空能,且忽略辐射。
\begin{equation}
    \begin{aligned}
    &\rho=\rho_{M}+\rho_{\Lambda}\\
    &P=P_{M}+P_{\Lambda}=-\rho_{\Lambda}\\
    &\rho+3 P = 0 \\
    & \Rightarrow \rho_{\Lambda}=\frac{1}{2} \rho_{M}
    \end{aligned}
\end{equation}

\begin{equation}
    K=\frac{8}{3} \pi G \rho a_{E}^{2}=8\pi G \rho_\Lambda a_{E}^{2}>0
\end{equation}

所以是正曲率,$K=+1$,

\begin{equation}
    a_E = 1/\sqrt{8\pi G \rho_\Lambda}
\end{equation}

但这个解不稳定。我们做微扰:
\begin{eqnarray}
    a&=&a_{E}+\delta a \\ 
    \rho_{M}&=&2 \rho_\Lambda+\delta \rho \\ 
\end{eqnarray}

要总满足
\begin{equation}
    1=\frac{8}{3} \pi G \rho a^{2}
\end{equation}

\begin{equation}
    \delta a<0 \Rightarrow \delta \rho>0 
\end{equation}
使得 $\dot{a}=0$ 继续满足,但二阶导

\begin{equation}
    \frac{3 \ddot{a}}{a}=-4 \pi G(\rho+3 P)=-4 \pi G\left(3 \rho_\Lambda+\delta \rho-3 \rho_\Lambda\right)=-4 \pi G \delta \rho<0
\end{equation}

$\delta a<0  \Rightarrow \ddot{a}<0$,
$\delta a>0  \Rightarrow \ddot{a}>0$,
即静态模型不稳定。

\end{document}
